
\subsection*{Abstract}\label{abstract}
\addcontentsline{toc}{subsection}{Abstract}

Conservation auctions can been used to meet the objectives of natural resource managers and are designed to cost-efficiently protect biodiversity and ecosystem services. In running an auction, an agency seeks to minimize the cost it pays to get the maximum benefit from the assets bought in the auction. When the auction is completed, cost (the total cost of the winning bids) tends to be known with certainty. Often however, benefits are more uncertain (e.g., when the benefits are to be realized in the future, such as in the case of vegetation regeneration). Therefore, there is an imperative on the part of the auction-running agency to spend resources on learning about benefits  to make a decision about which bids will be successful and which bids will be unsuccessful. Here we use the concept of expected value of sample information to assess how such an agency should allocate a limited learning budget among bids to conduct a conservation auction that yields the most cost-efficient return on investment. We propose a simple model system where an agent has two or more assets among which they must pick the most cost-efficient to invest in for a forthcoming auction. The cost-efficiency of each asset is more or less uncertain and there is some budget allocated to reduce the uncertainty of the assets. The agent must decide how to allocate the learning budget among the assets in order to maximize the expected benefit of the auction. We describe an analytical solution to the problem when there are two assets and the uncertainty in cost-efficiency is normally distributed. When there are more the two assets in the auction pool, then the analytical solution does not apply. Instead, we propose a heuristic solution to the problem based on optimally ranking the assets in order of cost-efficiency (rather than knowing explicitly, the exact value that determines the ranks). We compare the results of the heuristic solution to a Monte Carlo simulation of the problem to demonstrate that it arrives at the same optimal allocation of learning resources. In applying our model to case-studies with different numbers of assets and with different prior knowledge of the assets cost-efficiency, we glean a number of rules-of-thumb that may be helpful to agencies conducting conservation auctions. When learning budgets are small then the most optimal strategy is to learn new information about the most marginal assets, that is, the assets about which the probability of being the most cost-efficient is the most uncertain. Then, as the budget increases it becomes more optimal to learn about assets which have uncertain cost-efficiency in the more general sense. Finally, as learning budgets become large enough as to not be limiting, resources can be allocated evenly across all uncertain assets. The model we propose and the analytical, heuristic solutions we apply to its case studies, imply that naive even allocation (or no allocation at all) to learning among assets in a conservation auction may lead to less than cost-efficient outcomes for the agencies that use them.
