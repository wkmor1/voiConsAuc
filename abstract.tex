
\subsection*{Abstract}\label{abstract}
\addcontentsline{toc}{subsection}{Abstract}

Conservation auctions are tools for natural resource managers to cost-efficiently protect biodiversity and ecosystem services. Benefits particularly induce uncertainty in the value to be maximised. Therefore learning about benefits is valuable to decide which bids will be successful. Here we use the concept of expected value of sample information to assess how such an agency should allocate a limited learning budget among bids to conduct a conservation auction that yields the most cost-efficient return on investment. We propose a simple model where an agent has two or more assets among which they must pick the single most cost-efficient to invest in with an auction. The cost-efficiency of each asset is more or less uncertain and there is some budget allocated to reduce the uncertainty of the assets. The agent must decide how to allocate the learning budget among the assets to maximize the expected benefit of the auction. We describe an analytical solution to the problem when there are two assets and the uncertainty in cost-efficiency is normally distributed. When there are more than two assets in the auction pool, then the analytical solution does not apply. Instead, we propose a heuristic solution to the problem based on optimally ranking the assets in order of cost-efficiency. We compare the results of the heuristic solution to a Monte Carlo simulation of the problem to demonstrate that it arrives at the same optimal allocation of learning resources. We glean a number of rules-of-thumb that may be helpful to agencies conducting conservation auctions. When learning budgets are small, then it is best to learn about the most marginal assets. Then, as the budget increases it becomes more optimal to learn about assets which have uncertain cost-efficiency in the more general sense. Finally, as learning budgets become large enough, resources can be allocated evenly across all uncertain assets. Our findings imply that a naive even allocation to learning among assets in a conservation auction may lead to less than cost-efficient outcomes.
